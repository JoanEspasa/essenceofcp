\section{Introduction}

Most probably, during your career as a programmer you have found
a seemingly simple problem that, when it increased a bit in size, you could not
get a solution for it in a reasonable amount of time. All the ideas and approaches
you had for it seemed useless and you wondered why.

And if you still have not, I guarantee that at some point you will.
% because they are everywhere!
This text aims to make you a more effective and knowledgeable programmer,
by providing a new and shiny tool to confront them. But first at all, lets give 
these problems a name.

\subsection{Combinatorial Problems}

A \emph{combinatorial problem} consists in finding, among a finite set of
objects, one that satisfies a set of constraints. With Constraint Programming
you will be able to solve many kinds of combinatorial problems. 

Amongst combinatorial problems, the ones we are specially interested in are the
ones for which exhaustive search is not tractable. For solving these problems
we will need to identify the patterns or regularities in them and exploit these
in a clever way; to make deductions and decisions until we make our way into a
solution that satisfies all the constraints. All of this without introducing any
bug and considering all edge cases.

And as a programmer, you know that the effort in doing all of the above is not
small. Luckly for us we are not the first ones to confront these problems.

\subsection{What is Constraint Programming?}

The majority of programming languages were designed with the imperative
paradigm in mind. In these languages, statements describe how the program state
changes.  The work of the programmer is to devise which sequences of statements
produce the desired result.

In contrast, Constraint Programming (CP) is a declarative paradigm. Now instead
of focusing on \emph{how}, the programmer should focus on stating \emph{what}
the program should accomplish. This kind of paradigm will allow us to translate
a problem statement into a solvable problem specification without having to
come up with an imperative algorithm.  If we were to situate Constraint
Programming in a map, it would be a picturesque isle between the seas of
artificial intelligence, computer science and operations research.

Constraint Programming states relations between variables in the form of
constraints, specifying the properties of the solution. These constraints can
take many forms, such as logical ($\neg$,$\vee$ or $\wedge$) or numerical (+,
-, $*$) operators for example.

The set of constraints is then solved by giving a value to each variable so
that the solution is consistent with the stated constraints.  Do not worry, we
will give more formal and complete definitions further in the book, but for now
it should suffice.

\subsection{Modelling}

In our setting, a \emph{model} is a translation from the description of a
combinatorial problem into a computable formulation. Models are typically
expressed in terms of a language that can be understood by an application
called a \emph{solver}. With a model, the solver will then perform analysis, 
inference and search on top of it. This will end in solutions to the model that
can then be interpreted in terms of the original description of the problem. 

%So now we have a language provided by the solver, and a problem to be solved.
%What else do we need? Expressing things is not always easy

\subsection{Essence Prime and Savile Row}

Savile Row is a modelling assistant for Constraint Programming. It provides a
high-level language for the user to specify their constraint problem, and
automatically translates that language to the input language of a constraint
solver. It is a flexible tool, so it is very easy to add new rules and new
translation pipelines.

Savile Row takes in the Essence Prime constraint modelling language. Essence
Prime is intended to be a declarative, solver-independent and high-level language. 

Savile Row is named after a street in London with many bespoke tailors. The
name comes from the idea of ``tailoring'' a constraint model in a variety of
different ways.


Savile Row supports various solvers and solver types.
CP
SAT
SMT

