As our first example, imagine that you are emigrating to a new country. For
your journey, the airline will be limiting the total weight of your luggage to
50kg.  To make sure you do not forget anything, you made a list of all the
items that you will need. Sadly their total weight surpasses the allowed limit,
so you will need to decide which elements you leave behind.

\begin{figure}
\begin{lstlisting}
language ESSENCE' 1.0 @\label{line-language}@
letting weights=[10, 30, 25, 12, 5, 15, 28] @\label{line-weights}@
\end{lstlisting}
    \caption{Our \emph{instance}: The contents of the file \texttt{my\_luggage.param}, listing all weights}
\label{fig-knapsack1param}
\end{figure}

In this scenario the smart thing to do is to use all the available weight. So
you start by making a list of the elements and their weight to figure out what
is the best combination. Figure~\ref{fig-knapsack1param} shows how we can
express this list in Essence Prime.

\begin{itemize}
    \item The first line is expressing the language and version that the
        information is expressed in. In this case, it is Essence Prime version
        1.0. 
    \item The second line lists the weights of our objects. The
        \texttt{letting} keyword lets us define a new variable named
        \texttt{weights}, which is a list of integers.  We know it is a list
        because the values are wrapped in square brackets.
\end{itemize}

Now that we have recorded the weights of all the elements we want carry with us,
we can model the problem very easily with Essence Prime. Figure~\ref{fig-knapsack1}
shows how this could be done.

\begin{figure}
\begin{lstlisting}
language ESSENCE' 1.0
letting weight_limit be 50

given weights : matrix indexed by [int(0..objects)] of int(1..) @\label{line-1dmatrix}@
find choosen : matrix indexed by [int(0..objects)] of bool

such that
(sum i : int(0..objects) . choosen[i] * weights[i]) = weight_limit
\end{lstlisting}
    \caption{Our \emph{problem}: The contents of the \texttt{luggage.eprime} file, with a model able to decide what to bring with us.}
\label{fig-knapsack1}
\end{figure}

\begin{itemize}
    \item In Line~\ref{line-1dmatrix} \dots
\end{itemize}        

Note that we have made a clear distinction between the two parts of the problem.
The data for our problem, or the \emph{instance}, is defined in one file named
\textt{my\_luggage.param}. The constraints will reside in the \emph{problem} file,
named \textt{luggage.eprime}. The data and constraints could both reside in the
\texttt{luggage.eprime} file, but by separating them it avoids repetition.
Imagine that your partner is also emigrating with you, and therefore
has to take the same kind of decision. It is much cleaner to have just another instance
instead of copying over all the constraints.


\begin{itemize}
\item explain all the parts of the models
\item explain intuitively how the solver approaches the solution of the problem
\item show how to execute SR and get a solution
\item consider the full knapsack problem by considering the monetary value of each object
and maximizing the cost of everything we select while respecting the weigth limit
\end{itemize}

\section{Basic building blocks}
 what is an instance or a problem
How to define a variable and a constraint

\section{Solving a Problem}
Pick a toy problem to demonstrate everything
Invoking SR from the command line

\section{Optimising a Solution}
optimisation


% calcudoku might be a nice twist on the classic sudoku puzzle
%https://newdoku.com/include/print.php?n=3&lang=en&op=1&nd=1

